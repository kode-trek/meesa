\documentclass[twocolumn]{article}
\usepackage[top=.5in, bottom=.5in, left=.75in, right=.75in]{geometry}
\usepackage{amsthm,amssymb,amsmath}
\usepackage{graphicx}
\usepackage{hyperref}
\title{The mkPDF: a review on some writing tools}
\author{
Amir vSpades \\ \url{vegaspades@gmail.com}
\and
M.S.Fadakee \\ \url{m.fadaki@in.iut.ac.ir}
}
\date{}
\begin{document}
\maketitle
\begin{abstract}
In this article, we compare HTML, LibreOffice Writer and LaTeX in order to find
out which one could be the most promising tool in terms of user-friendly,
typeset mathematics handling, package management and media support. On top of
that, we introduce mkPDF, a tool for preparing LaTeX documents.
\end{abstract}
\section{Introduction}
\LaTeX is a high-quality typesetting system which includes features designed for
the production of technical and scientific documentation[1]. \LaTeX, as a
markup language, similar to \textit{HTML}, uses commands (as tags in HTML) to
enhance plain text types to more advanced level documents, usually in static PDF
formats. Figure 1 illustrates an example of such conversion.
\newline
Figure 1: The writer uses plain text commands as opposed to the WYSIWYG word
processors.
\newline
In this study, we compare the following typesetting systems and softwares:
\begin{enumerate}
\item \textit{HTML}: The Hypertext Markup Language (HTML) is the standard markup
language for documents designed to be displayed in a web browser.
\item \textit{LibreOffice Writer}: A free and open-source word processor
component of the LibreOffice software package, similar to Microsoft Word[2].
\item \textit{TeX/LaTeX}: A high-quality typesetting system; that includes
features designed for the production of technical and scientific
documentation[3].
\end{enumerate}
to find out which one could be the most promising tool in terms of:
\begin{enumerate}
\item \textit{User Friendly}: Readily enough for a novice to get used to the
environment for producing their own documents.
\item \textit{Typeset Mathematics}: Convenient enough to produce beautifully
complex typeset mathematics expressions.
\item \textit{Customizable}: Coherent package system that makes it relatively
easy for users to write extension packages for providing additional features.
\item \textit{Portability and Media Support}: Enables users for converting
documents to other outputs (mostly static PDF), allowing them to share and
publish them. Also supporting numerous different types of media within a
document.
\end{enumerate}
In addition, we introduced a humble software named as \textbf{mkPDF}, which uses
\textit{TeX-Live} compiler to:
\begin{enumerate}
\item Prepares documents using the TeX/LaTeX
\item Converts a <.tex> document to PDF
\item Formats lists of references using the BibTeX tool
\item Manually installs CTAN packages
\item Provides a sample <.tex> file that could be helpful for providing future
documents.
\end{enumerate}
The mkPDF is a CLI (Command Line Interface) tiny app that can be used by other
programs as in-line commands.
\newline
Figure 2: Enlisting the arguments and capabilities of mkPDF.
\newline
In section 2, we have a comparison among them three tools in terms of the
mentioned aspects. The scores would be A, B or C for Good, OK and Poor
respectively. The outcome is integrated and diagrammed via a bar chart.
\newline
In section 3, we look at some features and capabilities of the mkPDF app.
\section{Variables Evaluation}
For a quick comparison of them three tools, we gave them scores from A to C. The
result has been integrated and diagrammed via a bar chart.
\newline
Figure 3: a comparison on three tools, based on four variables.
\newline
The rest of this section provides details for this quick research.
\subsection{User Friendly}
\subsubsection{HTML (Score: A)}
One doesn't have to be a professional web-designer to publish a document in an
HTML format. The HTML <tags> are used to reform a plain text.
\newline
\textbf{Example 1.} Using the <b> and <\\b> tags to bold a phrase:
\newline
... blah blah blah ...
$\rightarrow$
... blah <b>blah<\\b> blah ...
\newline
The output would be like:
\newline
... blah \textbf{blah} blah ...
\newline
Nevertheless there are plenty of softwares to Edit HTML documents, namely:
\begin{enumerate}
\item Adobe Dreamweaver
\item Eclipse
\item Atom
\item ...
\end{enumerate}
Those kinds of IDEs, make tasks as WYSIWYG ( What You See Is What You Get)
thanks to them Shortcut Keys and Macros.
\subsubsection{LibreOffice Writer (Score: A)}
The LibreOffice Writer is another example for a WYSIWYG word processor.
\newline
\textbf{Example 2.} With LibreOffice Writer we could bold a phrase by
highlighting that phrase and pressing <Ctrl+B>:
\newline
... blah blah blah ...
$\rightarrow$
... blah blah blah ...
\newline
The output would be like:
\newline
... blah \textbf{blah} blah ...
\subsubsection{\LaTeX (Score: A)}
On the other hand, when writing with LaTeX, the writer uses plain text as
opposed to the formatted text found in WYSIWYG word processors. The writer uses
commands, as tags in HTML, to reformat the raw text.
\newline
\textbf{Example:} We may use
\begin{verbatim}
\textbf{...}
\end{verbatim}
command to make a phrase in bold typeface:
... blah blah blah ...
$\rightarrow$
... blah
\begin{verbatim}
\textbf{blah}
\end{verbatim}
blah ...
\newline
The output would be like:
\newline
... blah \textbf{blah} blah ...
\newline
Despite an initially steep learning curve, a novice can very quickly get to the
point where they are comfortable producing their own documents and can look up
answers to many of their questions online.
\newline
Most \LaTeX programs for editing <.tex> document files have a very user-friendly
interface:
\begin{enumerate}
\item \textit{Overleaf}: An online \LaTeX editor
\item \textit{TeXstudio}: An integrated writing environment for creating LaTeX
documents.
\item ...
\end{enumerate}
\subsection{Typeset Mathematics Handling}
\subsubsection{LibreOffice Writer (Score: A)}
The \textit{LibreOffice Writer} has a feature to quickly generate math
expression.
\textbf{Example:} All we have to do is highlighting the expression, and clicking
on the icon button:. LibreOffice Writer tries to recognize and interprete
commands inside that expression and if possible convert that syntax to a
good-looking math symbols.
\newline
The output woud be like:
\newline
\subsubsection{HTML (Score: B)}
HTML doesn't seem to be designed specefically for diplaying math expressions.
Anyway we could insert math symbols among the text using their equavelant hex
codes.
\newline
\textbf{Example: Pigeonhole principle.} Here there are n = 10 pigeons in m = 9
holes. Since 10 is greater than 9, the pigeonhole principle says that at least
one hole has more than one pigeon:
\newline
$⌈n/m⌉$
\newline
The hex codes to indicate such ceiling functions $⌈⌉$ are as follows:
\newline
\begin{verbatim}
&#x2308; n/m &#x2309;$
\end{verbatim}
Also we may get the same result using named HTML Entity:
\newline
\begin{verbatim}
&lceil; n/m &rceil;
\end{verbatim}
\subsubsection{LaTeX (Score: A)}
After all, LaTeX is designed by mathematicians for producing beautifully typeset
mathematics. LaTeX is famouse for rendering equations and mathematical symbols.
\textbf{Example: Pigeonhole Principle.} Here there are n = 10 pigeons in m = 9
holes. Since 10 is greater than 9, the pigeonhole principle says that at least
one hole has more than one pigeon:
$\lceil n/m \rceil$
\newline
The commands to indicate such ceiling functions $\lceil \ \rceil$ are:
\newline
\begin{verbatim}
$\lceil n/m \rceil$
\end{verbatim}
\subsection{Customizable}
\subsubsection{LibreOffice (Score: B)}
LibreOffice Extensions are tools that can be added or removed independently from
your installation of the main program.
\footnote{They are availabe at: https://extensions.libreoffice.org/}
They are feasible to download and install, while most of them seems to be fancy
and not necessary.
\subsubsection{HTML (Score: A)}
With the combiniation of (HTML, CSS, JS), one could build powerful open-source
extensions for a web browser to boost up its ability and performance. For
example to elicit information or ban annoying advertisments.
\footnote{For the Mozilla Firefox browser a variety of themes and extensions
could be found at:
https://support.mozilla.org/en-US/kb/find-and-install-add-ons-add-features-to-firefox}
\subsubsection{LaTeX (Score: C)}
While there are billions packages are available to reform raw text into a
beatiful document, there is one major problem known as \textbf{Dependency Hell}.
It describes as :
\newline
``Dependency hell is a colloquial term for the frustration of some software
users who have installed software packages which have dependencies on specific
versions of other software packages.''
\newline
\textbf{Example:} Installing the xe-persian package for \LaTeX deponds on the
following chained packages:
\newline
$> > > > ...$
\newline
On the contrary, the \textit{MiKTeX} can be configured in such a way that
missing packages are automatically installed.
\footnote{https://docs.miktex.org/manual/texfeatures.html}
\subsection{Portability and Media Support}
\subsubsection{HTML (Score: A)}
The combinition of (HTML, CSS, JS) does one exceptional job at producing
interactive webpages where there is no limit in displaying any formats of media.
It's so easy to export the outcome document in a variaty of formats, such as
PDF, HTML, etc.
\newline
\subsubsection{LibreOffice Writer (Score: B)}
Also LibreOffice Writer support embedding variaety of media formats within a
documnet, including: images and even videos. However inserting GIF (animated
pictures) might be challenging.
\subsection{Conclusion}
From this study, we established that while \LaTeX is the best choice for
preparing static documents (especially if they contain mathematics symbols), the
HTML combining with CSS and JS are well suited for making interactive pages.
\newline
In addition we evaluated a CMD app, named as mkPDF, in order to build static
papers in PDF format out of <.tex> documents.
\newline
References
\newline
\begin{verbatim}
[1] https://www.latex-project.org
\newline
[2] https://en.wikipedia.org/wiki/LibreOffice_Writer
\newline
[3] https://en.wikipedia.org/wiki/LaTeX
\newline
http://www.fileformat.info/info/unicode/char/2309/index.htm
\newline
http://www.fileformat.info/info/unicode/char/2308/index.htm
\end{verbatim}
\end{document}
